\documentclass{article}
\usepackage[catalan]{babel}
\usepackage{array,multirow}
\usepackage{hyperref}
\usepackage{amsfonts,amssymb,amsmath,amsthm, wasysym}
\usepackage{listings}
\usepackage[T1]{fontenc}        
\usepackage[utf8]{inputenc}


%Definitions of numbers
\newcommand\RR{\Bbb{R}}
\newcommand\QQ{\Bbb{Q}}
\newcommand\CC{\Bbb{C}}
\newcommand\ZZ{\Bbb{Z}}
\newcommand\NN{\Bbb{N}}


\usepackage{threeparttable}
\usepackage[text={15cm,18.5cm},centering]{geometry}


\begin{document}


%\renewcommand{\thefootnote}{\Alph{footnote}}

\newcommand{\pp}[1]{P(\#1\right)}
\newcommand{\vegeu}[1]{{(\footnotesize vegeu (\#1))}}
\newcounter{cas}
\newcounter{aux}
\renewcommand{\thecas}{\Roman{cas}}
\newcommand{\posacas}{\addtocounter{cas}{1}{\bf \thecas}}


\begin{center}
\textsc{Matemàtiques II dels graus de Biologia i Bioquímica.\\[0.5ex]  Taules de contrastos d'hipòtesi més usuals II: Dues mostres}
\end{center}
\vspace*{2cm}

\noindent En este documento recogemos los contrastes de hipótesis paramétricos más usuales para dos muestras  que se pueden llevar a cabo ``a mano.'' Para cada contraste donam: las condiciones, el estadístico
de contraste, la región crítica, 
el intervalo de confianza y el p-valor.

En la definición de los el estadísticos hemos empleado la notaciones siguientes:
\begin{itemize}
\item  $Z$: {Distribució normal estàndard $N(0,1)$.} 
\item $t_n$: {Distribució
$t$ de Student amb $n$ graus de llibertat.} 
\item $\chi_n^2$: {Distribució
khi-quadrat amb $n$ graus de llibertat.} 
\item $F_{n_1,n_2}$: {Distribució $F$ de
Fisher  amb $n_1$ i $n_2$ graus de llibertat.}
\item $X_\alpha$: Indica el $\alpha$-cuantil de la variable aleatoria $X$, es decir (si $X$ es continua, que es siempre el caso en este documento), el valor donde la función
de distribución de vale $X$  $\alpha$: $P(X\leq X_\alpha
)=\alpha$. 

Recordáis la traducción a los cuantiles de las propiedades de simetría de , $Z$ $t$ y :. $F$
\begin{itemize}
\item Simetría de la normal: $z_\alpha = -z_{1-\alpha}.$
\item Simetría de la $t$ de Student: $t_{n,\alpha} = -t_{n,1-\alpha}.$
\item Permutación de los grados de libertad de la $F$ de Fisher: $F_{n_1,n_2,\alpha}=\frac{1}{F_{n_2,n_1,1-\alpha}}.$
\end{itemize}
\end{itemize}
Los contrastes paramétricos con R los estudiam a la lección 23 del manual. 

\begin{center}
\small
\renewcommand{\arraystretch}{1.3}
\begin{tabular}{|c|c|c|c|c|}
\hline
\multicolumn{5}{|c|} {\bf Tipo de contraste y condiciones}\\ 
\hline
Hip.\ nuł.la Condiciones &&Muestra&
Hip.\ alto.&Caso\\
\hline\hline
\multirow{9}{2.5 cm}{$H_0:\mu_1 =\mu_2$\\ Cas independent}&\multirow{3}{3 cm}{$\sigma_1$ i $\sigma_2$ conegudes. Poblacions normals o $n_1$ i $n_2$ grans.} &
\multirow{3}{3 cm}{Dues m.a.s.\ independents de mides $n_1$ i $n_2$}&
$H_1:\mu_1\not =\mu_2$&  
\posacas\\\cline{4-5} & & & $H_1:\mu_1<\mu_2$&\posacas\\\cline{4-5}
 & & &$H_1:\mu_1>\mu_2$&\posacas\\\cline{2-5} & \multirow{3}{3cm}{$\sigma_1$ i $\sigma_2$ desconegudes i $\sigma_1 =\sigma_2$.\\ 
 Poblacions normals o $n_1$ i $n_2$ grans.}&
 \multirow{3}{3cm}{Dues m.a.s.\ independents de mides $n_1$ i $n_2$}&
 $H_1:\mu_1\not =\mu_2$&\posacas\\\cline{4-5}
 & & & $H_1:\mu_1<\mu_2$&\posacas\\\cline{4-5}
 & & & $H_1:\mu_1>\mu_2$&\posacas\\\cline{2-5} & \multirow{3}{3 cm}{$\sigma_1$ i $\sigma_2$ desconegudes i $\sigma_1 \neq\sigma_2$.\\ 
 Poblacions normals o $n_1$ i $n_2$ grans.}&\multirow{3}{3cm}{Dues m.a.s.\ independents de mides $n_1$ i $n_2$}&
$H_1:\mu\not =\mu_2$&\posacas\\\cline{4-5} & & &$H_1:\mu_1<\mu_2$&\posacas\\
\cline{4-5} & & &$H_1:\mu_1>\mu_2$&\posacas\\\hline
\multirow{9}{2.5 cm}{$H_0:\mu_1 =\mu_2$\\ Cas dependent}&
\multirow{3}{3 cm}{Dues poblacions normals dependents o $n$ gran.
$\sigma_d$
coneguda.${}^{\mathrm{(1)}}$}
&\hspace*{-3.2ex}\multirow{3}{2.5cm}{Dues m.a.s.\ dependents de mida $n$}&
$H_1:\mu_1\not =\mu_2$&\posacas\\\cline{4-5} & & & $H_1:\mu_1 <\mu_2$&\posacas\\
\cline{4-5} & & & $H_1:\mu_1 >\mu_2$&\posacas\\\cline{2-5} & 
\multirow{3}{3cm}{Dues poblacions normals dependents. $\sigma_d$
desconeguda.${}^{\mathrm{(1)}}$}&
\hspace*{-3.2ex}\multirow{3}{2.5cm}{Dues m.a.s.\ dependents de~mida $n$}&$H_1:\mu_1\not
=\mu_2$& 
\posacas\\\cline{4-5} & & &$H_1:\mu_1< \mu_2$&\posacas
\\\cline{4-5} & & &$H_1:\mu_1>\mu_2$&\posacas
\\\cline{2-5} & 
\multirow{3}{3cm}{Dues poblacions dependents, $n$ gran. $\sigma_d$
desconeguda.${}^{\mathrm{(1)}}$}&
\hspace*{-3.2ex}\multirow{3}{2.5cm}{Dues m.a.s.\ dependents de mida $n$}&$H_1:\mu_1\not
=\mu_2$& 
\posacas\\\cline{4-5} & & &$H_1:\mu_1< \mu_2$&\posacas
\\\cline{4-5} & & &$H_1:\mu_1>\mu_2$&\posacas\\
\hline
\multirow{3}{2.5cm}{$H_0:p_1 =p_2$\\ Cas independent}&
\multirow{3}{3cm}{Poblacions Bernoulli, $n_1$ i $n_2$ grans, molts èxits i fracasos.}&
\multirow{3}{3cm}{Dues m.a.s.\ independents de mides $n_1$ i $n_2$}&
$H_1 :p_1\not =p_2$&\posacas\\\cline{4-5} & & &
$H_1 :p_1 <p_2$&\posacas\\\cline{4-5}& & &
$H_1 :p_1 >p_2$&\posacas\\\hline
\multirow{3}{2.5cm}{$H_0:p_a=p_d$\\ Cas dependent}&
\multirow{3}{3cm}{Poblacions Bernoulli, $n_1$ i $n_2$ grans, molts casos discordants.}&
\multirow{3}{3cm}{Dues m.a.s.\ dependents de mida $n$}&
$H_1:p_a\not = p_b$&\posacas\\\cline{4-5} & & &
$H_1:p_a < p_b$&\posacas\\\cline{4-5}& & &
$H_1:p_a > p_b$&\posacas\\\hline
\multirow{3}{2.5cm}{$H_0:\sigma_1^2=\sigma_2^2$\\ Cas independent}&
\multirow{3}{3cm}{Poblacions normals.}&\multirow{3}{3cm}{Dues m.a.s.\ independents de mides $n_1$ i $n_2$}
&$H_1:\sigma_1^2\not =\sigma_2^2$&\posacas\\\cline{4-5} & & & 
$H_1:\sigma_1^2 <\sigma_2^2$&\posacas\\\cline{4-5} & & &
$H_1:\sigma_1^2 >\sigma_2^2$&\posacas\\\hline
%\multirow{3}{2.5cm}{/$H_0:\sigma_1^2=\sigma_2^2/$\\ Cas dependent}& 
%\multirow{3}{3cm}{Poblacions normals.}&\multirow{3}{3cm}{Dues m.a.s.\ dependents de mida $n$}&
%$H_1:\sigma_1^2\not =\sigma_2^2/$&\posacas\\\cline{4-5} & & &
%$H_1:\sigma_1^2 <\sigma_2^2$&\posacas\\\cline{4-5} & & &
%/$H_1:\sigma_1^2 >\sigma_2^2/$&\posacas\\\hline
\end{tabular}
\begin{itemize}
\item[(1)] $\sigma_d$ es la desviación típica de la variable 
$D=X_1 -X_2$.
\end{itemize}
\end{center}


\setcounter{cas}{0}
\begin{center}
\renewcommand{\arraystretch}{1.3}
\hspace*{-1.5cm}\begin{tabular}{|>{\small}c@{}|c@{\hspace*{-4ex}}|>{$\scriptstyle}c<{$}@{}|
@{}>{$\scriptstyle}c<{$}|@{}>{$\scriptstyle}c<{$}@{\hspace*{-1ex}}|}
\hline
\multicolumn{5}{|c|} {\bf Detalles de la maceta}\\ 
\hline
Caso&Estadístico&\mbox{Regió
crítica}&\mbox{Interval confiança}&\mbox{$p$-valor}\\\hline\hline
\posacas&\multirow{3}{2.5cm}{$Z=
\frac{\overline{X}_1-\overline{X}_2}{\widetilde{S}}$\\ és  $N(0,1)$\\  \vegeu{a}}& \{Z\leq
-z_{1-\frac{\alpha}{2}}\}\cup \{Z \geq z_{1-\frac{\alpha}{2}}\}&
\left]\overline{X}_1 -\overline{X}_2
-z_{1-\frac{\alpha}{2}}\widetilde{S},
\overline{X}_1 -\overline{X}_2
+z_{1-\frac{\alpha}{2}}\widetilde{S}\right[ & 2P(Z \geq |z|)\\
\cline{1-1}\cline{3-5}\posacas & &\{Z\leq z_{\alpha}\}&
\left]-\infty,
\overline{X}_1 -\overline{X}_2
-z_{\alpha}\widetilde{S}\right[ & P(Z \leq z)\\
\cline{1-1}\cline{3-5}
\posacas & &\{Z\geq z_{1-\alpha}\}&\left]
\overline{X}_1 -\overline{X}_2
-z_{1-\alpha}\widetilde{S},+\infty\right[ & P(Z\geq z)\\\hline
