\documentclass{article}
\usepackage[catalan]{babel}
\usepackage{array,multirow}
\usepackage{hyperref}
\usepackage{amsfonts,amssymb,amsmath,amsthm, wasysym}
\usepackage{listings}
\usepackage[T1]{fontenc}        
\usepackage[utf8]{inputenc}


%Definitions of numbers
\newcommand\RR{\Bbb{R}}
\newcommand\QQ{\Bbb{Q}}
\newcommand\CC{\Bbb{C}}
\newcommand\ZZ{\Bbb{Z}}
\newcommand\NN{\Bbb{N}}
\renewcommand{\leq}{\leqslant}
\renewcommand{\geq}{\geqslant}

\usepackage{threeparttable}
\usepackage[text={16cm,19.5cm},centering]{geometry}


\begin{document}


%\renewcommand{\thefootnote}{\Alph{footnote}}

\newcounter{cas}
\newcounter{aux}
\renewcommand{\thecas}{\Roman{cas}}
\newcommand{\posacas}{\addtocounter{cas}{1}{\bf \thecas}}

\begin{center}
\textsc{Tablas de contrastes de hipótesis más usuales I: una muestra}
\end{center}
\vspace*{2cm}

\noindent En este documento recogemos los contrastes de hipótesis paramétricos más usuales para una muestra  que se pueden llevar a cabo ``a mano.'' Para cada contraste damos: las condiciones, el estadístico
de contraste, la región crítica, 
el intervalo de confianza y el p-valor.

En la definición de los estadísticos hemos usado las notaciones siguientes:
\begin{itemize}
\item  $Z$: {Distribución normal estándard $N(0,1)$.} 
\item $t_n$: {Distribución
$t$ de Student con $n$ grados de libertad.} 
\item $\chi_n^2$: {Distribución
khi-cuadrado con $n$ grados de libertad.} 
\item $X_\alpha$: Indica el $\alpha$-cuantil de la variable aleatoria $X$, es decir (si $X$ es continua, que es siempre el caso en este documento), el valor donde la función
de distribución de $X_\alpha$ vale $\alpha$: $P(X\leq X_\alpha
)=\alpha$. 

Recordemos las propiedades de simetría de $Z$ y   $t$:
\begin{itemize}
\item Simetría de la normal: $z_\alpha = -z_{1-\alpha}.$
\item Simetría de la $t$ de Student: $t_{n,\alpha} = -t_{n,1-\alpha}.$
\end{itemize}
\end{itemize}

\begin{center}
\renewcommand{\arraystretch}{1.7}
\begin{tabular}{|c@{\hspace{1mm}}|c|c|c|c@{\hspace{1mm}}|}
\hline
\multicolumn{5}{|c|} {\bf Tipo de contraste y condiciones}\\ 
\hline
Hipótesis nula & Condiciones &Muestra&\multicolumn{1}{m{2 cm}}{
Hipótesis alternativa}&\multicolumn{1}{|c|}{Caso}\\
\hline\hline
\multirow{9}{2.5 cm}{$H_0:\mu =\mu_0$} &\multirow{3}{3 cm}{Población normal o $n$ grande.\\ $\sigma$ conocida. } &  
\multirow{3}{3cm}{$n$ observaciones independientes.} & 
$H_1:\mu\not =\mu_0$&
\posacas \\\cline{4-5} & & &$H_1:\mu <\mu_0$&\posacas \\\cline{4-5}
 & & &$H_1:\mu>\mu_0$&\posacas\\\cline{2-5} & \multirow{3}{3 cm}{Población normal.\\ $\sigma$ desconocida.
} & \multirow{3}{3cm}{$n$ observaciones independientes.}&
$H_1:\mu\not =\mu_0$&\posacas \\\cline{4-5} & & &$H_1:\mu <\mu_0$&\posacas\\
\cline{4-5} & & & $H_1:\mu>\mu_0$&\posacas
\\\cline{2-5} & \multirow{3}{3 cm}{Población cualquiera.\\ $\sigma$ desconocida.\\
 $n$ grande. } & \multirow{3}{3cm}{$n$ observaciones independientes.}&
$H_1:\mu\not =\mu_0$&\posacas \\\cline{4-5} & & &$H_1:\mu <\mu_0$&\posacas\\
\cline{4-5} & & & $H_1:\mu>\mu_0$&\posacas
\\\hline
\multirow{3}{2.5cm}{$H_0:p=p_0$}&\multirow{3}{3cm}{%
Población Bernoulli.
$n\geq 100$, $n\widehat{p}\geq 10$, $n(1-\widehat{p})\geq 10$
}&\multirow{3}{3cm}{$n$ observaciones independientes.}
&$H_1:p\not =p_0$&\posacas\\\cline{4-5} & & &
$H_1:p <p_0$&\posacas\\\cline{4-5}& & &
$H_1:p >p_0$&\posacas\\\hline
\multirow{3}{2.5cm}{$H_0:\sigma^2 =\sigma_0^2$}&
\multirow{3}{3cm}{Población Normal.\\ $\mu$ desconocida}&
\multirow{3}{3cm}{$n$ observaciones independientes.}&
$H_1:\sigma^2\not =\sigma_0^2$&\posacas\\\cline{4-5} & & &
$H_1:\sigma^2 <\sigma_0^2$&\posacas\\\cline{4-5} & & &
$H_1:\sigma^2 >\sigma_0^2$&\posacas\\\hline
\end{tabular}
\end{center}

\setcounter{cas}{0}

\begin{center}
\renewcommand{\arraystretch}{1.25}
\begin{tabular}{|>{\small}c@{}|c@{}|>{$\scriptstyle}c<{$}@{}|
@{}>{$\scriptstyle}c<{$}|@{}>{$\scriptstyle}c<{$}|}
\hline
\multicolumn{5}{|c|} {\bf Detalles del contraste}\\ 
\hline
Caso&Estadístico&\multicolumn{1}{c|}{Región
crítica}&\multicolumn{1}{c|}{Intervalo confianza}&\multicolumn{1}{c|}{$p$-valor}\\\hline\hline
\posacas&\multirow{3}{2cm}{$Z=
\frac{\overline{X}-\mu_0}{\frac{\sigma}{\sqrt{n}}}$\\  es $N(0,1)$}&\{Z\leq
-z_{1-\frac{\alpha}{2}}\}\cup \{Z \geq z_{1-\frac{\alpha}{2}}\}&
\left]\overline{X}-z_{1-\frac{\alpha}{2}}\cdot \frac{\sigma}{\sqrt{n}},
\overline{X}+z_{1-\frac{\alpha}{2}}\cdot \frac{\sigma}{\sqrt{n}}\right[ & 2 P(Z\geq |z|) \\
\cline{1-1}\cline{3-5}\posacas & &\{Z\leq z_{\alpha}\}&
\left]-\infty,
\overline{X}-z_{\alpha}\cdot \frac{\sigma}{\sqrt{n}}\right[ & P(Z \leq z)\\
\cline{1-1}\cline{3-5}
\posacas & &\{Z\geq z_{1-\alpha}\}&\left]
\overline{X}-z_{1-\alpha}\cdot \frac{\sigma}{\sqrt{n}},\infty\right[& P(Z \geq z)\\\hline
\posacas&\multirow{3}{2cm}{$T=
\frac{\overline{X}-\mu_0}{\frac{\tilde{S}}{\sqrt{n}}}$\\
es $t_{n-1}$}&\{T\leq
-t_{n-1,1-\frac{\alpha}{2}}\}\cup \{T \geq t_{n-1,1-\frac{\alpha}{2}}\}&
\left]\overline{X}-t_{n-1,1-\frac{\alpha}{2}}\cdot \frac{\tilde{S}}{\sqrt{n}},
\overline{X}+t_{n-1,1-\frac{\alpha}{2}}\cdot 
\frac{\tilde{S}}{\sqrt{n}}\right[ & 2P(t_{n-1} \geq |T|) \\\cline{1-1}\cline{3-5}\posacas & &
\{T\leq t_{n-1,\alpha}\}&
\left]-\infty,
\overline{X}-t_{n-1,\alpha}\cdot \frac{\tilde{S}}{\sqrt{n}}\right[ & P(t_{n-1} \leq T)\\
\cline{1-1}\cline{3-5}
\posacas & &\{T\geq t_{n-1,1-\alpha}\}&\left]
\overline{X}-t_{n-1,1-\alpha}\cdot
\frac{\tilde{S}}{\sqrt{n}},\infty\right[& P(t_{n-1}\geq T)
\\\hline
\posacas&\multirow{3}{2cm}{$Z =
\frac{\overline{X}-\mu_0}{\frac{\tilde{S}}{\sqrt{n}}}$\\
es aprox.\ $N(0,1)$}&\{Z\leq
-z_{1-\frac{\alpha}{2}}\}\cup \{Z \geq z_{1-\frac{\alpha}{2}}\}&
\left]\overline{X}-z_{1-\frac{\alpha}{2}}\cdot \frac{\tilde{S}}{\sqrt{n}},
\overline{X}+z_{1-\frac{\alpha}{2}}\cdot
\frac{\tilde{S}}{\sqrt{n}}\right[  & 2 P(Z \geq |z|)\\\cline{1-1}\cline{3-5}\posacas & &
\{Z\leq z_{\alpha}\}&
\left]-\infty,
\overline{X}-z_{\alpha}\cdot \frac{\tilde{S}}{\sqrt{n}}\right[ & P(Z \leq z)\\
\cline{1-1}\cline{3-5}
\posacas & &\{Z\geq z_{1-\alpha}\}&\left]
\overline{X}-z_{1-\alpha}\cdot
\frac{\tilde{S}}{\sqrt{n}},\infty\right[ & P(Z \geq z)
\\\hline
\posacas
&\multirow{3}{2.5cm}{
$Z=\frac{\widehat{p}-p_0}{\sqrt{\frac{p_0 (1-p_0)}{n}}}$\\ es $N(0,1)$} &\{Z\leq z_{\frac{\alpha}{2}}\}\cup 
\{Z\geq z_{1-\frac{\alpha}{2}}\} &\left]\widehat{p}+z_{\frac{\alpha}{2}}\sqrt{
\frac{\widehat{p}(1-\widehat{p})}{n}},\widehat{p}+z_{1-\frac{\alpha}{2}}\sqrt{
\frac{\widehat{p}(1-\widehat{p})}{n}}\right[
& 2P(Z \geq |z|)
\\\cline{1-1}\cline{3-5}
\posacas&
&\{Z\leq z_{\alpha}\}&\left]-\infty,\widehat{p}-z_{\alpha}\sqrt{
\frac{\widehat{p}(1-\widehat{p})}{n}}\right[ & P(Z\leq z)\\\cline{1-1}\cline{3-5}
\posacas&
&\{Z\geq z_{1-\alpha}\}&\left]\widehat{p}-z_{1-\alpha}\sqrt{
\frac{\widehat{p}(1-\widehat{p})}{n}},\infty\right[ & P(Z \geq z) \\
\hline
\posacas\footnotemark
 &\multirow{3}{2.5cm}{${\chi^2} =\frac{(n-1)\tilde{S}^2}{\sigma_0^2}/$\\ es $\chi^2_{n-1}/$}&\{\chi^2\leq
\chi_{n-1,\frac{\alpha}{2}}^2\}\cup
\{\chi^2\geq \chi_{n-1,1-\frac{\alpha}{2}}^2\}&
\left]\frac{(n-1)\tilde{S}2}{\chi_{n-1,1-\frac{\alpha}{2}}^2},
\frac{(n-1)\tilde{S}^2}{\chi_{n-1,\frac{\alpha}{2}}^2}\right[ & 
\begin{tabular}[c]{>{$\scriptstyle}c<{$}}
2\min\{P(\chi_{n-1}^2 \leq \chi^2), \\
P(\chi_{n-1}^2\geq \chi^2\}
\end{tabular}\\
\cline{1-1}\cline{3-5}\posacas & & \{\chi^2\leq \chi_{n-1,\alpha}^2\} &
\left]0,\frac{(n-1)\tilde{S}^2}{\chi_{n-1,\alpha}^2}\right[ & P(\chi_{n-1}^2 \leq \chi^2)\\
\cline{1-1}\cline{3-5}\posacas & & \{\chi^2\geq \chi_{n-1,1-\alpha}^2\} &
\left]\frac{(n-1)\tilde{S}2}{\chi_{n-1,1-\alpha}^2},\infty\right[ & P(\chi_{n-1}^2\geq \chi^2)\\\hline
\end{tabular}
\end{center}

\footnotetext{En este caso (\textbf{XIII}),  si $\mu$ es conocida, se puede usar el estadístico
$\chi^2 =\frac{\sum\limits_{i=1}^n {(X_i -\mu)}^2}{\sigma_0^2}$, 
que tendrá distribución $\chi_n^2$.}

\end{document}
